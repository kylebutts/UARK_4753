\documentclass[12pt]{article}
\usepackage{../../lecture_notes}
\usepackage{../../math}
\usepackage{../../uark_colors}

\hypersetup{
  colorlinks = true,
  allcolors = ozark_mountains,
  breaklinks = true,
  bookmarksopen = true
}


\providecommand{\tightlist}{%
  \setlength{\itemsep}{0pt}\setlength{\parskip}{0pt}}\usepackage{longtable,booktabs,array}
\usepackage{calc} % for calculating minipage widths
% Correct order of tables after \paragraph or \subparagraph
\usepackage{etoolbox}
\makeatletter
\patchcmd\longtable{\par}{\if@noskipsec\mbox{}\fi\par}{}{}
\makeatother
% Allow footnotes in longtable head/foot
\IfFileExists{footnotehyper.sty}{\usepackage{footnotehyper}}{\usepackage{footnote}}
\makesavenoteenv{longtable}
\usepackage{graphicx}
\makeatletter
\def\maxwidth{\ifdim\Gin@nat@width>\linewidth\linewidth\else\Gin@nat@width\fi}
\def\maxheight{\ifdim\Gin@nat@height>\textheight\textheight\else\Gin@nat@height\fi}
\makeatother
% Scale images if necessary, so that they will not overflow the page
% margins by default, and it is still possible to overwrite the defaults
% using explicit options in \includegraphics[width, height, ...]{}
\setkeys{Gin}{width=\maxwidth,height=\maxheight,keepaspectratio}
% Set default figure placement to htbp
\makeatletter
\def\fps@figure{htbp}
\makeatother

\usepackage{float}
\usepackage{tabularray}
\usepackage[normalem]{ulem}
\usepackage{graphicx}
\UseTblrLibrary{booktabs}
\UseTblrLibrary{siunitx}
\NewTableCommand{\tinytableDefineColor}[3]{\definecolor{#1}{#2}{#3}}
\newcommand{\tinytableTabularrayUnderline}[1]{\underline{#1}}
\newcommand{\tinytableTabularrayStrikeout}[1]{\sout{#1}}
\makeatletter
\@ifpackageloaded{caption}{}{\usepackage{caption}}
\AtBeginDocument{%
\ifdefined\contentsname
  \renewcommand*\contentsname{Table of contents}
\else
  \newcommand\contentsname{Table of contents}
\fi
\ifdefined\listfigurename
  \renewcommand*\listfigurename{List of Figures}
\else
  \newcommand\listfigurename{List of Figures}
\fi
\ifdefined\listtablename
  \renewcommand*\listtablename{List of Tables}
\else
  \newcommand\listtablename{List of Tables}
\fi
\ifdefined\figurename
  \renewcommand*\figurename{Figure}
\else
  \newcommand\figurename{Figure}
\fi
\ifdefined\tablename
  \renewcommand*\tablename{Table}
\else
  \newcommand\tablename{Table}
\fi
}
\@ifpackageloaded{float}{}{\usepackage{float}}
\floatstyle{ruled}
\@ifundefined{c@chapter}{\newfloat{codelisting}{h}{lop}}{\newfloat{codelisting}{h}{lop}[chapter]}
\floatname{codelisting}{Listing}
\newcommand*\listoflistings{\listof{codelisting}{List of Listings}}
\makeatother
\makeatletter
\makeatother
\makeatletter
\@ifpackageloaded{caption}{}{\usepackage{caption}}
\@ifpackageloaded{subcaption}{}{\usepackage{subcaption}}
\makeatother

\begin{document}
\begin{center}
  {\Huge\bf Homework 1}
  
  \smallskip
  {\large\texttt{[ECON 4753]} — \textit{University of Arkansas}}

  % \medskip
  % {\large Prof. Kyle Butts}
\end{center}

\section*{Review Questions}

\subsection*{Question 1}

\begin{enumerate}[label=(\alph*)]
  \item What does $\sum_{i = 1}^5 (i - 3)$ equal?

  \item Calculate the sample mean, $\bar{x} = \frac{1}{n} \sum_{i = 1}^n x_i$, where the sample observations are $x = (2, 7, 10, 6, 8)$ 
\end{enumerate}

\subsection*{Question 2}

This question is based on our review of statistics. Say you observe a
sample of workers from a firm with sample size \(n = 100\). You observe
their wages \(w_i\) and want to estimate the average wage at the firm.
You estimate the following statistics in your sample:
\(\bar{w} = 17.53\) and \(\var{w} = 4.2\).

\medskip
\begin{enumerate}[label=(\alph*)]
  \item Given this information what is the (approximate) sample distribution of the sample mean?

  \item Form a 95\% confidence interval for your sample mean. Interpret this in words.

  \item Another student claims the average worker earns \$17. 
  Using your confidence interval, would you reject this null with a 5\% significance level?
\end{enumerate}

\newpage
\section*{R Coding}

This assignment will explore a sample of homes in Boston suburbs. It
comes from the paper
\href{sciencedirect.com/science/article/pii/0095069678900062}{Hedonic
housing prices and the demand for clean air} which tries to estimate how
much people are willing to pay to live in homes with cleaner air.

To use this dataset, use the function \texttt{read.csv} with the url
\url{https://raw.githubusercontent.com/kylebutts/UARK_4753/main/Homework/HW1/data/housing_df.csv}.
Remember that you need to include the code that loads the dataset into
your R Markdown file.

This data set has the following variables:

\begin{table}[h!]
\centering
\begin{tblr}[         %% tabularray outer open
]                     %% tabularray outer close
{                     %% tabularray inner open
colspec={Q[]Q[]},
}                     %% tabularray inner close
\toprule
Variable & Info \\ \midrule %% TinyTableHeader
MEDV & Median value of owner-occupied homes in \$1000's \\
CRIM & Per capita crime rate by town \\
ZN & Proportion of residential land zoned for lots over 25,000 sq. ft. \\
INDUS & Proportion of non-retail business acres per town \\
CHAS & Charles River indicator (=1 if census tract touches river) \\
NOX & Nitric oxides concentration (parts per 10 million) \\
RM & Average number of rooms per dwelling \\
AGE & Proportion of owner-occupied units built priort to 1940 \\
DIS & weighted distances to five Boston employment centres \\
RAD & index of accessibility to radial highways \\
TAX & Full-value property-tax rate per \$10,000 \\
PTRATIO & Pupil-teacher ratio by town \\
B & Formula involving \% Black \\
LSTAT & \% lower status of the population \\
\bottomrule
\end{tblr}
\end{table}

\subsection*{Question 1}

Look in the \texttt{Environment} panel of R studio, how many variables
and how many observations are in this data set?

\subsection*{Question 2}

Each observation is a town in the suburbs or Boston. First, we want to
get a sense of the distribution of the \emph{median} value of
owner-occupied homes of towns in the Boston suburb at this time.

Use the function \texttt{mean()} and \texttt{sd()} to find the average
median value of owner-occupied homes in \$1000's. Report the R code and
number to two digits.

\subsection*{Question 3}

To practice with R coding. Use the functions \texttt{sqrt()},
\texttt{sum()}, \texttt{\^{}2}, \texttt{length()}, and \texttt{mean()},
but not \texttt{var()} or \texttt{sd()}, calculate the sample standard
deviation median value of owner-occupied homes in \$1000's. Report the R
code and number to two digits.

\subsection*{Question 4}

Use the \texttt{hist()} function, create a histogram of NOX pollution.
Give this graph a nice title.

\subsection*{Question 5}

Now with a sense of the distribution of the two variables of interest.
Let's make a scatter plot of \texttt{MEDV} on the x-axis and
\texttt{NOX} on the y-axis. You can use the
\texttt{plot(x\ =\ \_\_\_\_,\ y\ =\ \_\_\_\_)} function for this.
Include the plot in the output.

For this question, we are going to practice making high-quality reports.
When presenting our work to stakeholders, we want it to look good.

\begin{itemize}
  \item The stakeholders want readable axis titles. Use the function arguments \texttt{xlab = ""}, and \texttt{ylab = ""} to improve the axis labels.
  
  \item Our figure needs a title. Use the chunk option \texttt{fig.cap = ""} to describe the relationship between neighborhood NOX levels and home prices. 

  \item The stakeholders do not care about the code used to generate the figure. Let's hide it for this question using the chunk option \texttt{echo = FALSE}
\end{itemize}


\end{document}
