\documentclass[12pt]{article}
\usepackage{../../lecture_notes}
\usepackage{../../math}
\usepackage{../../uark_colors}

\hypersetup{
  colorlinks = true,
  allcolors = ozark_mountains,
  breaklinks = true,
  bookmarksopen = true
}


\providecommand{\tightlist}{%
  \setlength{\itemsep}{0pt}\setlength{\parskip}{0pt}}\usepackage{longtable,booktabs,array}
\usepackage{calc} % for calculating minipage widths
% Correct order of tables after \paragraph or \subparagraph
\usepackage{etoolbox}
\makeatletter
\patchcmd\longtable{\par}{\if@noskipsec\mbox{}\fi\par}{}{}
\makeatother
% Allow footnotes in longtable head/foot
\IfFileExists{footnotehyper.sty}{\usepackage{footnotehyper}}{\usepackage{footnote}}
\makesavenoteenv{longtable}
\usepackage{graphicx}
\makeatletter
\def\maxwidth{\ifdim\Gin@nat@width>\linewidth\linewidth\else\Gin@nat@width\fi}
\def\maxheight{\ifdim\Gin@nat@height>\textheight\textheight\else\Gin@nat@height\fi}
\makeatother
% Scale images if necessary, so that they will not overflow the page
% margins by default, and it is still possible to overwrite the defaults
% using explicit options in \includegraphics[width, height, ...]{}
\setkeys{Gin}{width=\maxwidth,height=\maxheight,keepaspectratio}
% Set default figure placement to htbp
\makeatletter
\def\fps@figure{htbp}
\makeatother

\makeatletter
\@ifpackageloaded{caption}{}{\usepackage{caption}}
\AtBeginDocument{%
\ifdefined\contentsname
  \renewcommand*\contentsname{Table of contents}
\else
  \newcommand\contentsname{Table of contents}
\fi
\ifdefined\listfigurename
  \renewcommand*\listfigurename{List of Figures}
\else
  \newcommand\listfigurename{List of Figures}
\fi
\ifdefined\listtablename
  \renewcommand*\listtablename{List of Tables}
\else
  \newcommand\listtablename{List of Tables}
\fi
\ifdefined\figurename
  \renewcommand*\figurename{Figure}
\else
  \newcommand\figurename{Figure}
\fi
\ifdefined\tablename
  \renewcommand*\tablename{Table}
\else
  \newcommand\tablename{Table}
\fi
}
\@ifpackageloaded{float}{}{\usepackage{float}}
\floatstyle{ruled}
\@ifundefined{c@chapter}{\newfloat{codelisting}{h}{lop}}{\newfloat{codelisting}{h}{lop}[chapter]}
\floatname{codelisting}{Listing}
\newcommand*\listoflistings{\listof{codelisting}{List of Listings}}
\makeatother
\makeatletter
\makeatother
\makeatletter
\@ifpackageloaded{caption}{}{\usepackage{caption}}
\@ifpackageloaded{subcaption}{}{\usepackage{subcaption}}
\makeatother

\begin{document}
\begin{center}
  {\Huge\bf Lab 8 Instructions}
  
  \smallskip
  {\large\texttt{[ECON 4753]} — \textit{University of Arkansas}}

  % \medskip
  % {\large Prof. Kyle Butts}
\end{center}

\section*{Time-series Analysis Project}

This assignment will summarize the second part of the course where we
studied time-series datasets. For this assignment, you will be asked to
perform a lightly guided analysis of your choice of one of 3 possible
datasets. I will give you some prompts to make sure you use the methods
we covered in the class.

You should think of this as a project where you are trying to conduct
inference about a time-series using the tools we have learned in class.

You will be graded, in part, on how professional your results look. All
graphs should be labelled clearly and you should write in complete
sentences in your writeup. I have provided a template that will hide
your code in your results document that you submit (in the real world,
your boss is not looking for your code).

\newpage
\section*{Semi-guided questions}

Please work through these questions and write up your results. See the
template for an example of how I want this to look.

\begin{enumerate}
  \item \emph{Exploring a Single Variable:} 
  
  Choose one continuous variable from the dataset and explore its distribution. What can you tell me about its distribution? Try visualizing the data using a histogram and/or calculate the mean and variance.
\end{enumerate}

\newpage
\section*{Datasets}

You have the choice of one of five datasets. You can see the brief
overview of each as well as the \emph{data dictionary} that describes
each variable in the dataset by clicking on the links.

\begin{enumerate}
\def\labelenumi{\arabic{enumi}.}
\item
  \href{https://github.com/kylebutts/UARK_4753/tree/main/Homework/HW2/data/airbnb_nashville}{AirBnB
  listings in Nashville} is a scrapped dataset from the website Inside
  Airbnb. It contains information about the listing as well as reviews
  received. I downloaded the current ``snapshot'' for Nashville,
  Tennessee.
\item
  The
  \href{https://github.com/kylebutts/UARK_4753/tree/main/Homework/HW2/data/nba_2k25_player_ratings}{NBA
  2K25} dataset contains information about all the players in the NBA
  2K25 video game. It contains info on the player (team, salary, years
  in nba) and their ratings.
\item
  \href{https://github.com/kylebutts/UARK_4753/tree/main/Homework/HW2/data/college_scorecard}{College
  Scorecard} is a dataset created by the US Education Department. The
  dataset was created to allow for potential college students to easily
  understand the cost-benefit of different colleges.
\item
  \href{https://github.com/kylebutts/UARK_4753/tree/main/Homework/HW2/data/imdb}{iMDB}
  collects user reviews of movies as well as some basic info of the
  film.
\item
  The
  \href{https://github.com/kylebutts/UARK_4753/tree/main/Homework/HW2/data/community_population_survey}{CPS}
  contains survey data on workers collected by the US Government. This
  has characteristics of workers and their wages.
\end{enumerate}


\end{document}
